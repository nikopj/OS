\documentclass[11pt,a4paper]{article}
\usepackage[utf8]{inputenc}
\usepackage{amsmath}
\usepackage{amsfonts}
\usepackage{amssymb}
\usepackage{graphicx}
\usepackage{verbatim}
\usepackage[margin=.8in]{geometry}
\usepackage{listings} % for formatting code
\lstset
{
    language=C,
    basicstyle=\footnotesize\ttfamily,
    numbers=left,
    stepnumber=1,
    showstringspaces=false,
    tabsize=4,
    breaklines=true,
    breakatwhitespace=false,
}

\setlength\parindent{0pt} % Removes all indentation from paragraphs

\author{Nikola Janju\v{s}evi\'{c}}
\title{ECE357, Computer Operating Systems -- Problem Set \#7}

\begin{document}
\begin{Large}
ECE357, Computer Operating Systems -- Problem Set \#7
\end{Large} \\
\begin{large}
Nikola Janju\v{s}evi\'{c}
\end{large} 
\\
\today

\section*{assembly process: \texttt{as\_show.sh}}
\lstinputlisting{../as_show.sh}

\section*{Problem 2 -- pure assembly}
\verb!$./as_show.sh p2.s!
\lstinputlisting[numbers=none]{../out2.txt}

\pagebreak

\section*{Problem 3 -- exit code}
The previous assembly code, \verb!p2.s!, did not make a call to the \verb!_exit! system call and was terminated by a segmentation fault as the processor attempted to continue to execute instructions beyond the defined text region, eventually attempting to access memory that was not currently mapped in (seen in \verb!si_code=SEGV_MAPERR!). The following code, \verb!p3.s!, uses the \verb!_exit! system call. \\\\
\verb!$./as_show.sh p3.s!
\lstinputlisting[numbers=none]{../out3.txt}
Note that both \verb!strace! and \verb!$?! verify that the program exited with the specified value, 2.

\section*{Problem 4 -- system call validation}
The following program calls \verb!syscall! with an invalid system call number. \\\\
\verb!$./as_show.sh p4.s!
\lstinputlisting[numbers=none]{../out4.txt}
\verb!strace! shows that the system call fails with \verb!errno=38!, which corresponds to failure due to an invalid system call number. The program still exits as per the specified \verb!_exit! system call.

\end{document}
